%!TEX root = ../GTNA.tex

\section{Plotting}
\label{sec:plotting}


\subsection{Multi}
\label{sec:plotting:multi}

\begin{verbatim}
Each series handled as a separate element, e.g.,
ER(N=1000,ad=10.0,b=true)
ER(N=1000,ad=15.0,b=true)
ER(N=1000,ad=20.0,b=true)
\end{verbatim}



\subsection{Single}
\label{sec:plotting:single}

\begin{verbatim}
Each series[] is combined as an element
Different parameter is used as value on the x-axis, e.g.,
ER(N=1000,ad=X,b=true) => x := Average Degree
ER(N=X,ad=25.0,b=true) => x := Nodes
\end{verbatim}




\subsection{Single-by}
\label{sec:plotting:single-by}

\begin{verbatim}
Each series[] is combined as an element
Specified parameter is used as value on the x-axis, e.g.,
DegreeDistribution - “EDGES”
ClusteringCoefficient - “CLUSTERING_COEFFICIENT”
\end{verbatim}





\subsection{Configuration}
\label{sec:plotting:configuration}







\subsection{Gnuplot scripts}
\label{sec:plotting:gnuplot}




\begin{verbatim}
Generation of plots using gnuplot wrapper
Multi-scalar plots
Single-scalar plots
Single-scalar plots by single-scalar
\end{verbatim}



\begin{lstlisting}[label={},caption={}]
public class Plotting {
public static boolean multi(Series[] s, Metric[] metrics, String folder,
			Type type, Style style) {...}

public static boolean single(Series[][] s, Metric[] metrics, String folder,
			Type type, Style style) {...}

public static boolean singleBy(Series[][] s, Metric[] metrics, String folder,
Metric metricX, String keyX,
Type type, Style style) {...}

...
}
\end{lstlisting}



